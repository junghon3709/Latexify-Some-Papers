\documentclass{article}
\usepackage[utf8]{inputenc}
\usepackage{amsmath, amssymb, tikz, graphics, biblatex, geometry, enumitem, float}
\DeclareMathOperator{\pro}{Proj}
\setlength{\parskip}{1em}
\setlength{\parindent}{0em}

\newcommand{\grstep}[2][\relax]{%
   \ensuremath{\mathrel{
       {\mathop{\longrightarrow}\limits^{#2\mathstrut}_{
                                     \begin{subarray}{l} #1 \end{subarray}}}}}}
\newcommand{\swap}{\leftrightarrow}

\title{MA2108S 17/18 Sem 2}
\author{Yip Jung Hon A0199560R}

\usepackage{lipsum}

\begin{document}
\maketitle
\subsection*{Question 1}
\begin{enumerate}[label=(\alph*)]
    \item $\sum_{n=4}^{\infty} \frac{1}{n \log(\sqrt{n} + \cos n) }$ diverges.
    \begin{align*}
        &\sum_{n=4}^{\infty} \frac{1}{n \log(\sqrt{n} + \cos n)} \geq \sum_{n=4}^{\infty} \frac{1}{n \log(\sqrt{n} + 1)}
    \intertext{By Cauchy Condensation Test, set $a_i = \frac{1}{n \log(\sqrt{n} + 1)}$, then $2^ia_{2^i} = \frac{2^i}{2^i \log(\sqrt{2^i} + 1)}= \frac{1}{\log(\sqrt{2^i} + 1)}$. The sum $ \sum_{n=4}^{\infty} \frac{1}{n \log(\sqrt{n} + 1)}$ will diverge if $\sum_{n=4}^{\infty} \frac{1}{\log(\sqrt{2^i} + 1)}$ diverges.}
    &\sum_{n=4}^{\infty} \frac{1}{\log(\sqrt{2^i} + 1)} \geq \sum_{n=4}^{\infty} \frac{1}{\log(\sqrt{2^i} + \sqrt{2^i})} = \sum_{n=4}^{\infty} \frac{1}{\log(2^{\frac{n}{2}+1})} = \frac{1}{\log(2)}\sum_{n=4}^{\infty} \frac{1}{(\frac{n}{2}+1)}
    \end{align*}
    which diverges.
    \item $\sum_{k=2}^\infty\frac{\sin k}{\log k}$ converges.
    The key point is that $\sum \sin k$ is bounded. Once you have $\sum \sin k$ bounded, apply Dirichlet's test and be done. To show $\sum \sin k$ is bounded, note that by factor formula,
    \begin{align*}
        2(\sin k)(\sin(0.5))=\cos(k-0.5)-\cos(k+0.5).
        \intertext{Summing up from $1 \cdots n$, we get a cancellation between adjacent terms. This implies that:}
        \sum_{k=1}^n \sin k = \frac{\cos(0.5)-\cos(n+0.5)}{2\sin(0.5)}
    \end{align*}
    which is clearly bounded. Fun fact: it is bounded by 5. 
    \item Good lord why do I hate myself. $\sum_{n=2}^{\infty} \frac{(-1)^{n-1}}{\sqrt{n} + (-1)^n}$ diverges. 
    Let's write the series out explictly.
    \begin{equation*}
        \sum_{n=2}^{\infty} \frac{(-1)^{n-1}}{\sqrt{n} + (-1)^n} = \frac{1}{\sqrt{3}-1} - \frac{1}{\sqrt{2}+1} + \frac{1}{\sqrt{5}-1} - \frac{1}{\sqrt{4}+1} + ...
    \end{equation*}
    Our series can be rewritten as:
    \begin{equation*}
        \frac{1}{\sqrt{3}-1} - \frac{1}{\sqrt{2}+1} + \frac{1}{\sqrt{5}-1} - \frac{1}{\sqrt{4}+1} + ... = \sum_{n=1}^{\infty} \frac{1}{\sqrt{2n+1}-1} - \frac{1}{\sqrt{2n}+1}
    \end{equation*}
    Now note that $\sqrt{2n}+1 \geq \sqrt{2n+1}$ for $n \in \mathbb{N}$. This can be easily seen by squaring both sides.
    \begin{align*}
        \sum_{n=1}^{\infty} \frac{1}{\sqrt{2n+1}-1} - \frac{1}{\sqrt{2n}+1} &\geq         \sum_{n=1}^{\infty} \frac{1}{\sqrt{2n+1}-1} - \frac{1}{\sqrt{2n+1}} \\
        &\geq \sum_{u \in \{3, 5, 7 \cdots \}}^{\infty} \frac{1}{\sqrt{u}-1} - \frac{1}{\sqrt{u}} \\
        &= \sum_{u \in \{3, 5, 7 \cdots \}}^{\infty} \frac{1}{u - \sqrt{u}}
    \end{align*}
    which diverges.
    \item $\sum_{n=2}^{\infty} \frac{(-1)^{n-1}}{n^{\alpha} + (-1)^n}$ converges if $\frac{1}{2}<\alpha$.
    As before, we may rewrite our sum as such,
    \begin{equation*}
        \sum_{n=2}^{\infty} \frac{(-1)^{n-1}}{n^{\alpha} + (-1)^n} = \sum_{n=1}^{\infty} \frac{1}{(2n+1)^{\alpha} - 1} - \frac{1}{(2n)^{\alpha}+1}.
    \end{equation*}
    If $\alpha=1$, we have,
    \begin{equation*}
        \sum_{n=1}^{\infty} \frac{1}{(2n+1)^{\alpha} - 1} - \frac{1}{(2n)^{\alpha}+1} = \sum_{n=1}^{\infty} \frac{1}{2n} - \frac{1}{2n+1} =\sum_{n=1}^{\infty} \frac{1}{4n^2+2n}
    \end{equation*}
    which converges. If $\frac{1}{2} < \alpha <1$,
    \begin{align*}
        \sum_{n=1}^{\infty} \frac{1}{(2n+1)^{\alpha} - 1} - \frac{1}{(2n)^{\alpha}+1} &\leq     \sum_{n=1}^{\infty} \frac{1}{(2n)^{\alpha} - 1} - \frac{1}{(2n)^{\alpha}+1} \\
        &\leq \sum_{n=1}^{\infty} \frac{2}{(2n)^{2\alpha}-1} 
    \end{align*}
    which converges based on our knowledge of the $p$-series.
\end{enumerate}

\subsection*{Question 2}
For convenience, $\lim_{n \to \infty}$ will just be abbreviated to $\lim$.
\begin{align*}
    \gamma &= \lim \left[ \sum_{i=1}^n \frac{1}{i} - \log{n} \right] \\
    &=  \lim \left[ \sum_{i=1}^n \frac{1}{i} - \sum_{i=1}^n \log \left( \frac{i+1}{i} \right) \right] \\
    &= \lim \left[ \sum_{i=1}^n \frac{1}{i} - \log \left( 1+ \frac{1}{i} \right) \right]
\end{align*}
However, $\frac{1}{i} - \log \left( 1+ \frac{1}{i} \right)>0$, which means that the sequence of partial sums is monotone increasing. The reason why this is so is from \textbf{Bernolli's inequality}, which says $e^x \geq 1+x$ for all $x \in \mathbb{R}$. Taking $\log$ on both sides and replacing $x$ with $\frac{1}{i}$ gives the desired result. Hence, we just need to show the sequence of partial sums is bounded above. Let $x= \frac{1}{i}$ and Taylor expand $\log(1+x)$. The Taylor expansion of $\log(1+x)$ is absolutely convergent on $-1<x<1$. \textbf{Note that $\log(1+x)$ has $x \leq 1$}, which means that this is a valid operation to do and it will not alter our infinite series.
\begin{align*}
    \lim \sum_{i=1}^n \left[x - \log \left( 1+ x \right)\right] 
    &=  \lim \sum_{i=1}^n  x-\left(x-\frac{x^2}{2!} + \frac{x^3}{3} - \frac{x^4}{4} + \cdots\right) \\ 
    &= \lim \sum_{i=1}^n  \left(\frac{x^2}{2} - \frac{x^3}{3} + \frac{x^4}{4} - \cdots \right) \\
    &= \lim \sum_{i=1}^n  \left(\frac{1}{2i^2} - \frac{1}{3i^3} + \frac{1}{4i^4} - \cdots \right)
    \intertext{We can subtract off smaller numbers by making the denominators bigger for the negative numbers.}
    &\leq \lim \sum_{i=1}^n  \left(\frac{1}{2i^2} - \frac{1}{4i^4} + \frac{1}{4i^4} - \cdots \right)\\
    &= \lim \sum_{i=1}^n \left( \frac{1}{2i^2} - \frac{1}{ni^n}\right)
    \intertext{With $\frac{1}{ni^n}$ vanishing if $n$ is even.}
    &\leq \lim \sum_{i=1}^n \left( \frac{1}{2i^2} \right) = \sum_{i=1}^{\infty} \left( \frac{1}{2i^2} \right) = \frac{1}{2} \frac{\pi^2}{6}
\end{align*}
and we are done.

\subsection*{Question 3}
Given
\begin{equation*}
    f(1) + \frac{x-1}{2} \leq f(x) + \frac{x-1}{3} \ \ \ : x \in [0, 1].
\end{equation*}
For any $x \in [0,1]$, one has:
\begin{align*}
    &\frac{x-1}{2} \leq f(x) - f(1)  \\
    &f(1) - f(x) \leq \frac{1-x}{3}
\end{align*}
Since $f(1)=1$ and $f(x_{n-1})=f(x_n)$, this implies,
\begin{equation*}
    |f(1)-f(x_n)| \leq \max\left\{\left| \frac{x-1}{2} \right|, \left| \frac{x-1}{3} \right|\right\} \leq \frac{1}{2} (f(1) -f(x_{n-1}))
\end{equation*}
We thus have this chain:
\begin{equation*}
    |f(1)-f(x_n)| \leq \frac{1}{2} |f(1)-f(x_{n-1}| \leq \cdots \leq \left( \frac{1}{2} \right)^n |f(1)-f(x_0)|
\end{equation*}
Given some $\epsilon>0$, there exists $N$ sufficiently big such that $\left( \frac{1}{2}\right)^N |f(1)-f(0)| < \epsilon$.
Then for all $n+1>N$, one has,
\begin{equation*}
    |f(1)-f(x_{n+1})| \leq \left(\frac{1}{2}\right)^{n+1} |f(1)-f(0)| < \left(\frac{1}{2}\right)^{n} |f(1)-f(0)| < \epsilon
\end{equation*}
and hence $|f(1)-f(x_{n+1})|=|1-x_n|<\epsilon$, giving us that $(x_n) \to 1$.




\end{document}
