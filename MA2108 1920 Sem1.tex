\documentclass{article}
\usepackage{geometry}
\geometry{a4paper}
\geometry{a4paper, left=25.4mm,top=25.4mm, right=25.4mm, bottom=25.4mm}
\usepackage{graphicx}
\usepackage{mathptmx}
\usepackage{enumerate}
\usepackage{pgfplots}
\pgfplotsset{compat=1.15}
\usepackage[parfill]{parskip}
\usepackage{amsmath, amsthm,amssymb}
\begin{document}
    {\LARGE{MA2108 AY19/20 Semester 1 Exam Suggested Solutions}}
    \vspace{0.2in}
    
    Author: Chong Jing Quan \hfill Reviewer: ????
    
    \par\noindent\rule{\textwidth}{0.4pt}
\section*{Question 1}
\begin{enumerate}[(a)]
    \item \begin{enumerate}[(i)]
        \item We prove this by induction. The case for $n=1$ is clear. Suppose the inequality holds for $n=k>1.$ We want to show that the inequality holds for $n=k+1.$ Indeed, we have $$x_{k+1}=\sqrt{x_k+6}\geq\sqrt{0+6}>0$$ and $$x_{k+1}=\sqrt{x_k+6}\leq\sqrt{3+6}=3,$$ which completes the induction step.
        \item We claim that the sequence converges to 3. Observe that $$|x_{n+1}-3|=|\sqrt{x_k+6}-3|=\left|\frac{x_k-3}{\sqrt{x_k+6}+3}\right|<\frac{1}{3}|x_k-3|.$$ Thus, the sequence contracts and so $\displaystyle\lim_{k\to\infty}x_k=3.$
    \end{enumerate}
    \item The answer is $\limsup y_n=1$ and $\liminf y_n=-1.$ We will only prove for limit superior case as the proof is similar for limit inferior. Note that
    $$\limsup y_n=\limsup \left(\frac{\cos n}{n}+\sin\frac{n\pi}{6}\right)\geq\limsup\left(\frac{\cos n}{n}\right)+\limsup\left(\sin\frac{n\pi}{6}\right)=0+1=1.$$
    On the other hand, suppose $\limsup y_n=1+\varepsilon$ for some $\varepsilon>0.$ Since for any $\varepsilon>0,$ it is possible to find a positive integer $N$ so that $\frac{1}{n}<\varepsilon$ for all positive integers $n\geq N,$ there exists a positive integer $K$ such that $$\frac{\cos (6k+3)}{(6k+3)}+\sin\left(\frac{6k+3}{6}\pi\right)\leq\frac{1}{6k+3}+1<1+\varepsilon$$ for all $k\geq K,$ which contradicts the definition of limit superior.
    \item Since $\sup a_n\geq a_n$ and $\inf b_n\leq b_n,$ we have $\dfrac{a_n}{b_n}\leq \dfrac{\sup a_n}{\inf b_n}.$ Since the inequality works for any positive integer $n,$ we get $\sup\dfrac{a_n}{b_n}\leq \dfrac{\sup a_n}{\inf b_n}.$ Taking limit on both sides gives $$\lim_{n\to\infty}\sup\frac{a_n}{b_n}=\limsup\frac{a_n}{b_n}\leq \lim_{n\to\infty}\dfrac{\sup a_n}{\inf b_n}=\dfrac{\lim_{n\to\infty}\sup a_n}{\lim_{n\to\infty}\inf b_n}=\frac{\limsup a_n}{\liminf b_n}$$ since $(a_n)$ and $(b_n)$ are bounded sequences.
\end{enumerate}
\section*{Question 2}
\begin{enumerate}[(a)]
    \item We have 
    \begin{eqnarray*}
    \sum_{n=1}^{\infty}\frac{2n+1}{n^2(n+1)^2}&=&\lim_{N\to\infty}\sum_{n=1}^{N}\frac{2n+1}{n^2(n+1)^2}\\
    &=&\lim_{N\to\infty}\sum_{n=1}^{N}\frac{(n+1)^2-n^2}{n^2(n+1)^2}\\
    &=&\lim_{N\to\infty}\sum_{n=1}^{N}\left(\frac{1}{n^2}-\frac{1}{(n+1)^2}\right)\\
    &=&\lim_{N\to\infty}\left(1-\frac{1}{(N+1)^2}\right)\ =\ 1.
    \end{eqnarray*}
    \item \begin{enumerate}[(i)]
        \item The series does not converge because $\dfrac{3n^3-2n^2+n+1}{5n^4-3n^3+2}>\dfrac{1}{5n}$ for positive integers $n$ and $\displaystyle\sum^{\infty}_{n=1}\frac{1}{n}$ diverges.
        \item The series converges by root test. We have $$\lim_{n\to\infty}\sqrt[n]{\frac{n^2}{10^n}\left(1+\frac{1}{2n}\right)^{4n^2}}=\lim_{n\to\infty}\frac{1}{10}n^{\frac{2}{n}}\left(1+\frac{1}{2n}\right)^{4n}.$$
        Since $\displaystyle\lim_{n\to\infty}n^{\frac{2}{n}}=1$ and $\displaystyle\lim_{n\to\infty}\left(1+\frac{1}{2n}\right)^{4n}=e^2<9,$ the required limit is less than 1 and so the series converges.
    \end{enumerate}
    \item Write $0\leq b_n-a_n\leq c_n-a_n.$ Since $\displaystyle\sum^{\infty}_{n=1}(c_n-a_n)$ converges (absolutely), the series $\displaystyle\sum^{\infty}_{n=1}(b_n-a_n)$ converges (absolutely) by comparison test. Hence, the series $\displaystyle\sum^{\infty}_{n=1}b_n=\displaystyle\sum^{\infty}_{n=1}(b_n-a_n)+\displaystyle\sum^{\infty}_{n=1}a_n$ converges too.
\end{enumerate}
\section*{Question 3}
\begin{enumerate}[(a)]
    \item Let $\varepsilon>0$ be given. Pick $\delta=\min\left\{1,\dfrac{\varepsilon}{12}\right\}$ so that $0<|x+2|<\delta\implies \left|\dfrac{2x-3}{2x+3}-7\right|<\varepsilon.$ Indeed, we have $$\left|\frac{2x-3}{2x+3}-7\right|=\left|\frac{-12x-24}{2x+3}\right|=12|x+2|\left|\frac{1}{2x+3}\right|<12|x+2|<12\times\frac{\varepsilon}{12}=\varepsilon.$$ The conclusion follows.
    \item The function is only continuous at $x=2.$ Let $\varepsilon>0$ be given. Take $\delta=\dfrac{\varepsilon}{3}$ so that $|f(x)-5|<\varepsilon.$ Indeed, we have $$ |f(x)-5|\leq \sup\{|3x-6|,|2x-4|\}=3|x-2|<3\times\frac{\varepsilon}{3}=\varepsilon.$$ Thus, the function is continuous at $x=2.$
    
    For $x\neq2,$ consider two cases. If $x$ is rational, then $f(x)=3x-1.$ Consider a sequence of irrational numbers $(x_n)^{\infty}_{n=1}$ that converges to $x.$ Then, $f(x_k)=2x_k+1$ for each positive integer $k.$ 
    
    Since $x\neq 2,$ the limit $\displaystyle\lim_{k\to\infty}(2x_k+1)=2x+1$ does not equal to $f(x)=3x-1.$ Thus, the function is not continuous at rational values other than 2. The case for $x$ is irrational can be handled similarly.
    \item \begin{enumerate}[(i)]
        \item Since $\displaystyle\lim_{n\to\infty}\frac{g(2x)}{g(x)}=1,$ for a given $\varepsilon,$ there exists a positive real number $N$ so that $\left|\dfrac{g(2x)}{g(x)}-1\right|<\varepsilon$ for all $x>N.$ Since $2^{n-1}x\geq x$ for positive integers $n,$ we have $\left|\dfrac{g(2^nx)}{g(2^{n-1}x)}-1\right|<\varepsilon$ and we are done.
        \item Notice that for $\alpha>2,$ we can write $\alpha=2^k\beta$ for some positive integer $k$ and real number $1\leq\beta<2.$ As such, we have $$\lim_{x\to\infty}\frac{g(\alpha x)}{g(x)}=\lim_{x\to\infty}\frac{g(2^k\beta x)}{g(x)}=\lim_{x\to\infty}\left(\frac{g(2^k\beta x)}{g(2^{k-1}\beta x)}\frac{g(2^{k-1}\beta x)}{g(2^{k-2}\beta x)}\cdots \frac{g(2\beta x)}{g(\beta x)}\frac{g(\beta x)}{g(x)}\right).$$
        A modification of the proof for part (i) yields $\displaystyle\lim_{x\to\infty}\frac{g(2^k\beta x)}{g(2^{k-1}\beta x)}=1.$ On the other hand, since $g$ is increasing, we have $g(x)<g(\beta x)<g(2x)$ and so $1<\dfrac{g(\beta x)}{g(x)}<\dfrac{g(2x)}{g(x)}.$ By squeeze theorem, the limit is $\displaystyle\lim_{x\to\infty}\frac{g(\beta x)}{g(x)}=1.$ Hence, we conclude that $$\lim_{x\to\infty}\frac{g(\alpha x)}{g(x)}=1.$$
    \end{enumerate}
\end{enumerate}
\section*{Question 4}
\begin{enumerate}[(a)]
    \item By Extreme Value Theorem, $f$ attains its supremum at $x_1\in[0,1]$ and $g$ attains its supremum at $x_2\in[0,1].$ If $x_1=x_2,$ nothing to prove.
    
    Suppose $f(x_1)>f(x_2)$ and $g(x_1)<g(x_2),$ i.e. $f$ and $g$ attains maximum at different points. Then, we see that $f(x_1)-g(x_1)=g(x_2)-g(x_1)>0$ and $f(x_2)-g(x_2)=f(x_2)-f(x_1)<0.$ Thus, by Intermediate Value Theorem, there exists $x_0\in[0,1]$ so that $f(x_0)=g(x_0).$
    \item Since $f$ is uniformly continuous, there exists $\delta>0$ so that $|x-y|<2\delta\implies|h(x)-h(y)|<1.$ Thus, if $|x|=k\delta+r$ for some positive integer $k$ and $0\leq r<\delta$ by triangle inequality, we get 
    \begin{eqnarray*}
    |h(x)-h(0)|&=&|h(x)-h(x-\delta)+h(x-\delta)-h(x-2\delta)+\cdots+h(r)-h(0)|\\
    &\leq&|h(x)-h(x-\delta)|+|h(x-\delta)-h(x-2\delta)|+\cdots+|h(r)-h(0)|\\
    &\leq&k+1.
    \end{eqnarray*}
    Thus, $|h(x)|\leq |h(x)-h(0)|+|h(0)|\leq k+1+|h(0)|.$ Since $|x|=k\delta+r\geq k\delta,$ it follows that $$|h(x)|\leq \dfrac{|x|}{\delta}+1+|h(0)|.$$ The proof is complete.
\end{enumerate}
\textbf{Comment.} In other words, from Question 4(b), a uniformly continuous function is at most $O(n)!$
\end{document}