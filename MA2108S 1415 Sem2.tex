\documentclass{article}
\usepackage[utf8]{inputenc}
\usepackage{amsmath, amssymb, tikz, graphics, biblatex, geometry, enumitem, float}
\DeclareMathOperator{\pro}{Proj}
\setlength{\parskip}{1em}
\setlength{\parindent}{0em}
\usepackage{xcolor}

\newcommand{\grstep}[2][\relax]{%
   \ensuremath{\mathrel{
       {\mathop{\longrightarrow}\limits^{#2\mathstrut}_{
                                     \begin{subarray}{l} #1 \end{subarray}}}}}}
\newcommand{\swap}{\leftrightarrow}
\usepackage[colorlinks=true,linkcolor=blue, citecolor=blue, allcolors=blue ]{hyperref}
\urlstyle{tt}

\title{MA2108S 14/15 Sem 2}
\author{Yip Jung Hon A0199560R}

\usepackage{lipsum}

\begin{document}
\maketitle
\subsection*{Question 1}
We want to show that for all $\epsilon>0$, there exists $N$ such that $\left|\frac{a_n^n}{n!}\right|<\epsilon$.
Consider the term
\begin{equation}
    \frac{a_{N+n}^{N+n}}{(N+n)!} = \frac{a_{N+n}^n}{(N+n)(N+n-1) \cdots (N+1)} \times \frac{a_{N+n}^N}{N!} \label{eq1}
\end{equation}
Intuitively, to make (\ref{eq1}) small, we want to find an $N$ such that $\frac{a_{N+n}^N}{N!}$ can be made arbitrarily small, and for all $a_{N+i}$, $1 \leq i \leq n, \frac{a_{N+i}}{N+i}<1$. Since $(a_n) \to \alpha$, there exists some $N'$ such that $n>N' \implies |a_n - \alpha|<\epsilon \implies \alpha - \epsilon < a_n < \epsilon + \alpha$. Since $a_n$ is bounded, this means there exist an $N_1$ sufficiently large such that $N_1!>\frac{a_{N_1}^{N_1}}{\epsilon}$. Similarly, there exists $N_2 > \alpha + \epsilon$. Set $N = \{N_1, N_2\}$. For all $n+N>N$, one has:
\begin{align*}
    \frac{a_{N+n}^n}{(N+n)(N+n-1) \cdots (N+1)} \times \frac{a_{N+n}^N}{N!} &\leq  \frac{(\alpha+\epsilon)^n}{(N+n) \cdots (N+1)} \times \frac{a_{N+n}^N}{N!} \\
    &< 1 \times \epsilon \\
    &< \epsilon
\end{align*}
Note that 
\begin{align*}
    |x_{n+2} -x_{n+1}| &= \left|\frac{3}{1+x_{n+1}} - \frac{3}{1+x_n} \right| \\
    &= \frac{3|x_{n+1}-x_n|}{(1+x_{n+1})(1+x_n)} \\
    &\leq \frac{3}{16} |x_{n+1}-x_n|
\end{align*}
Since the sequence is a contraction, it is Cauchy and thus has a limit.

\subsection*{Question 2}
Since the function is monotone increasing, $0 \leq f(b) -f(a)$. However, $f(b) \leq b$ and $f(a) \geq a$, $0 \leq f(b) -f(a) \leq b-a$. If the equality is strict, then $f$ is a contraction. By Banach's Fixed Point theorem, there exists an $x_0$ such that $f(x_0)=x_0$. Else, suppose $ f(b) -f(a) =b-a$. This would imply $f(b)=f(a)-a+b \geq b$. From the condition, $f(b) \leq b$, so $f(b)=b$.

% Since $(\gamma(x_n)) \to \gamma(b)$, there exists some $N$ such that $n>N \implies |\gamma(x_n)-\gamma(b)|<\epsilon$. Since $a \leq x_n \leq b$ and $\gamma$ is strictly increasing, $-\epsilon<\gamma(x_n)-\gamma(b)<0$. 

Suppose for contradiction that $\lim x_n \neq b$. That means that there exist some $\epsilon>0$ such that for all $N$ proposed, there exists $n>N$ such that $|x_n -b|\geq \epsilon$. This implies there are infinitely many points $|x_n -b|\geq \epsilon$. Since $x_n \leq b$, $|x_n -b|\geq \epsilon \implies x_n-b<-\epsilon$. Construct a subsequence $(y_n)$ of $(x_n)$ consisting of all terms $x_n-b<-\epsilon$. Then $\gamma(y_n) < \gamma(b)$, since $\gamma$ is strictly increasing and each $y_n$ is bounded by $b-\epsilon$. There is a positive difference between each $\gamma(y_n)$ and $\gamma(b)$, which means $\gamma(y_n)$ does not converge to $\gamma(b)$. Yet, this is a contradiction since $\gamma(y_n)$ is a subsequence of $\gamma(x_n)$.

\subsection*{Question 3}
We want to show the convergence of $$\sum_{n=0}^{\infty} \left(1+\frac{1}{2}+\frac{1}{3} + \cdots + \frac{1}{n}\right) \frac{\sin(\frac{n\pi}{2})}{n}$$.
First, $\sin(\frac{n\pi}{2})=1$ if $n$ mod 4 = 1, $\sin(\frac{n\pi}{2})=-1$ if $n$ mod 4 = 3, $\sin(\frac{n\pi}{2})=0$ if $n$ mod 4 = 0,2. The sum simplifies to:
\begin{equation*}
    \sum_{\text{$n$ odd}}^{\infty} \left(1+\frac{1}{2}+\frac{1}{3} + \cdots + \frac{1}{n}\right) \frac{\phi(n)}{n}
\end{equation*}
where $\phi(n)=1$ if $n$ mod 4 = 1, $\phi(n)=-1$ if $n$ mod 4 = 3 and $\phi(n)=0$ if $n$ mod 4 = 0,2. This sum will converge by the Alternating Series Test if we can show $\frac{1}{n}\left(1+\frac{1}{2}+\frac{1}{3} + \cdots + \frac{1}{n}\right)$ is monotone decreasing. We want to show:
\begin{equation*}
    \frac{1}{n} + \frac{1}{2n} + \cdots + \frac{1}{n^2} > \frac{1}{n+1} + \frac{1}{2(n+1)} + \cdots + \frac{1}{n(n+1)} + \frac{1}{(n+1)^2}
\end{equation*}
This can be done by moving terms to the left,
\begin{align*}
    \frac{1}{n} - \frac{1}{n+1} + \frac{1}{2n} - \frac{1}{2(n+1)} + \cdots + \frac{1}{n^2} - \frac{1}{(n+1)n} &> \frac{1}{(n+1)^2} \\
    \iff \frac{1}{n(n+1)} + \frac{1}{2(n)(n+1)} + \cdots \frac{1}{n \times n(n+1)} &> \frac{1}{(n+1)^2} \\
    \iff \frac{1}{n(n+1)} \left( 1+\frac{1}{2}+ \cdots + \frac{1}{n}\right) &> \frac{1}{(n+1)^2}
\end{align*}
The last statement is obviously true, since $\frac{1}{n(n+1)}> \frac{1}{(n+1)^2}$ and $\left( 1+\frac{1}{2}+ \cdots + \frac{1}{n}\right)>1$. Applying the Alternating Series Test shows convergence.

Since $\sum_{n=1}^\infty a_n$ converges, put $\sum_{n=1}^\infty a_n=b$. Then $(\sum_{n=1}^\infty a_n)^2=b^2$. From $a_n = a_{n+1}^2 + a_{n+2}^2 + \cdots$, 

\begin{alignat}{5} \nonumber
  &a_1&&= a_2^2 &&+ a_3^2 &&+ a_4^2 &&+ a_5^2  \cdots  \\ \label{eq2}
  &a_2&&= 0 &&+ a_3^2 &&+ a_4^2 &&+ a_5^2\cdots \\ \nonumber
  &a_3&&= 0 &&+ 0 &&+ a_4^2 &&+ a_5^2\cdots \\ \nonumber
  &a_4&&= 0 &&+ 0 &&+ 0 &&+ a_5^2\cdots \\ \nonumber
  & \ && \ && \ && \vdots && \  \nonumber
\end{alignat}
$b$ is the sum of all the non-zero terms here. However, a small modification can be made to the above: 
\begin{alignat*}{5}
  & && a_2^2 &&+ a_3^2 &&+ a_4^2 &&+ a_5^2  \cdots  \\
  & && \textcolor{red}{a_2^2} &&+ a_3^2 &&+ a_4^2 &&+ a_5^2\cdots \\
  & && \textcolor{red}{a_3^2} &&+ \textcolor{red}{a_3^2} &&+ a_4^2 &&+ a_5^2\cdots \\
  & && \textcolor{red}{a_4^2} &&+ \textcolor{red}{a_4^2} &&+ \textcolor{red}{a_4^2} &&+ a_5^2\cdots \\
  & \ && \ && \ && \vdots && \ 
\end{alignat*}
We may sub the red terms down the column (or diagonal) to see that what we have here is simply $2b$. The sum of all terms above converges to $2b$ and is absolutely convergent, meaning that one can rearrange the terms. $2b=2a_2^2 + 4a_3^2 + 6a_4^2 + 8a_5^2 + \cdots$. Setting the top row to $a_1$ in (\ref{eq2}) gives $b=a_1 + a_3^2 + 2a_4^2 + 3a_5^2+\cdots$.

\subsection*{Question 4}
\end{document}