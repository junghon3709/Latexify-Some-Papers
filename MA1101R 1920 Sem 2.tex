\documentclass{article}
\usepackage{geometry}
\geometry{a4paper}
\geometry{a4paper, left=25.4mm,top=25.4mm, right=25.4mm, bottom=25.4mm}
\usepackage{graphicx}
\usepackage{mathptmx}
\usepackage{enumerate}
\usepackage{pgf,tikz,pgfplots}
\pgfplotsset{compat=1.15}
\usepackage{mathrsfs}
\usetikzlibrary{arrows}
\usepackage[parfill]{parskip}
\usepackage{amsmath, amsthm,amssymb}
\newcommand{\spn}{\text{span}}
\newenvironment{rowequmat}[1]{\left(\array{@{}#1@{}}}{\endarray\right)}
\usepackage{mathtools}
\begin{document}
    {\LARGE{MA1101R AY19/20 Semester 2 Exam Suggested Solutions}}
    \vspace{0.2in}
    
    Author: Chong Jing Quan \hfill Reviewer: ????
    
    \par\noindent\rule{\textwidth}{0.4pt}
\section*{Question 1}
\begin{enumerate}[(a)]
    \item For any $w\in\spn\{\textbf{u}_1,\textbf{u}_2\}\cap\spn\{\textbf{v}_1,\textbf{v}_2,\textbf{v}_3\},$ we have $w=a_1\textbf{u}_1+a_2\textbf{u}_2=b_1\textbf{v}_1+b_2\textbf{v}_2+b_3\textbf{v}_3$ and so $$a_1\textbf{u}_1+a_2\textbf{u}_2-b_1\textbf{v}_1-b_2\textbf{v}_2-b_3\textbf{v}_3=0.$$
    Setting up the augmented matrix and row reducing yields
    \begin{eqnarray*}\begin{rowequmat}{ccccc|c}
     1 &  0 & 2 & 1 & 2&0\\
     0 &  -1 & -3 & 1& 2&0 \\
     -1 &  -2 & -4 & 2&3&0 \\
     -2 & 1 & -1 & -2&-3&0\\
     2 & 0 & 4 & 0&-2&0
\end{rowequmat}&\xrightarrow[-R_1+R_3]{2R_1+R_4}&\begin{rowequmat}{ccccc|c}
     1 &  0 & 2 & 1&2&0 \\
     0 &  -1 & -3 & 1&2&0 \\
     0 & -2 & -6 & 1&1&0\\
     0 & 1 & 3 & 0 &1&0\\
     2&0&4&0&-2&0
\end{rowequmat}\\
&\xrightarrow[-2R_1+R_5]{-R_2}&\begin{rowequmat}{ccccc|c}
     1 &  0 & 2 & 1&2&0 \\
     0 &  1 & 3 & -1&-2&0 \\
     0 & -2 & -6 & 1&1&0\\
     0 & 1 & 3 & 0 &1&0\\
     0&0&0&-2&-6&0
\end{rowequmat}\\
&\xrightarrow[2R_2+R_3]{-R_2+R_4}&\begin{rowequmat}{ccccc|c}
     1 &  0 & 2 & 1&2&0 \\
     0 &  1 & 3 & -1&-2&0 \\
     0 & 0 & 0 & -1&-3&0\\
     0 & 0 & 0 & 1 &3&0\\
     0&0&0&-2&-6&0
\end{rowequmat}\\
&\xrightarrow{-R_3}&\begin{rowequmat}{ccccc|c}
     1 &  0 & 2 & 1&2&0 \\
     0 &  1 & 3 & -1&-2&0 \\
     0 & 0 & 0 & 1&3&0\\
     0 & 0 & 0 & 1 &3&0\\
     0&0&0&-2&-6&0
\end{rowequmat}\\
&\xrightarrow[2R_3+R_5]{-R_3+R_4}&\begin{rowequmat}{ccccc|c}
     1 &  0 & 2 & 1&2&0 \\
     0 &  1 & 3 & -1&-2&0 \\
     0 & 0 & 0 & 1&3&0\\
     0 & 0 & 0 & 0 &0&0\\
     0&0&0&0&0&0
\end{rowequmat}\\
&\xrightarrow[R_3+R_2]{-R_3+R_1}&\begin{rowequmat}{ccccc|c}
     1 &  0 & 2 & 0&-1&0 \\
     0 &  1 & 3 & 0&1&0 \\
     0 & 0 & 0 & 1&3&0\\
     0 & 0 & 0 & 0 &0&0\\
     0&0&0&0&0&0
\end{rowequmat}
\end{eqnarray*}
Thus, a solution to the system is given by $\spn\left\{\begin{pmatrix}-2\\-3\\1\\0\\0\end{pmatrix},\begin{pmatrix}1\\-1\\0\\-3\\1\end{pmatrix}\right\}.$ It is now easy to see that $\textbf{v}_1$ and $-3\textbf{v}_2+\textbf{v}_3$ are in both $\spn\{\textbf{u}_1,\textbf{u}_2\}$ and $\spn\{\textbf{v}_1,\textbf{v}_2,\textbf{v}_3\}.$ Since $\textbf{v}_1$ and $-3\textbf{v}_2+\textbf{v}_3$ are linearly independent of each other, it follows that $S=\{\textbf{v}_1,-3\textbf{v}_2+\textbf{v}_3\}$ is one such set.
\item We aim to solve the system
\begin{eqnarray*}
(a_1\textbf{v}_1+a_2\textbf{v}_2+a_3\textbf{v}_3)\cdot \textbf{u}_1\ =\ 16a_1+3a_2+a_3&=&0\\
(a_1\textbf{v}_1+a_2\textbf{v}_2+a_3\textbf{v}_3)\cdot \textbf{u}_2\ =\ 10a_1-7a_2-11a_3&=&0.
\end{eqnarray*}
Setting up the augmented matrix and row reducing yields
\begin{eqnarray*}
\begin{rowequmat}{ccc|c}
     16 &  3 & 1 &0\\
     10 &  -7 & -11 &0 
\end{rowequmat}&\xrightarrow{-\frac{5}{8}R_1+R_2}&\begin{rowequmat}{ccc|c}
     16 &  3 & 1 &0\\
     0 &  -\frac{71}{8} & -\frac{93}{8} &0 
\end{rowequmat}\\
&\xrightarrow{-\frac{8}{71}R_2}&\begin{rowequmat}{ccc|c}
     16 &  3 & 1 &0\\
     0 &  1 & \frac{93}{71} &0 
\end{rowequmat}\\
&\xrightarrow{-3R_2+R_1}&\begin{rowequmat}{ccc|c}
     16 &  0 & -\frac{208}{71} &0\\
     0 &  1 & \frac{93}{71} &0 
\end{rowequmat}\\
&\xrightarrow{\frac{1}{16}R_1}&\begin{rowequmat}{ccc|c}
     1 &  0 & -\frac{13}{71} &0\\
     0 &  1 & \frac{93}{71} &0 
\end{rowequmat}.
\end{eqnarray*}

Thus, the system has a solution given by $\spn\left\{\begin{pmatrix}13\\-93\\71\end{pmatrix}\right\}.$ Thus, one such vector is given by $13\textbf{v}_1-93\textbf{v}_2+71\textbf{v}_3=(75,10,-25,-40,-90).$
\end{enumerate}
\section*{Question 2}
\begin{enumerate}[(a)]
    \item Using Gram-Schmidt process, we have
    \begin{eqnarray*}
    \textbf{u}_1&=&(5,2,6,-4)\\
    \textbf{u}_2&=&(-12,-3,-12,6)-\frac{(-12,-3,-12,6)\cdot(5,2,6,-4)}{5^2+2^2+6^2+(-4)^2}(5,2,6,-4)\\
    &=&(-2,1,0,-2)\\
    \textbf{u}_3&=&(2a+3,8a+3,-3a+6,2a-6)-\frac{(2a+3,8a+3,-3a+6,2a-6)\cdot(5,2,6,-4)}{5^2+2^2+6^2+(-4)^2}(5,2,6,-4)\\
    &-&\frac{(2a+3,8a+3,-3a+6,2a-6)\cdot(-2,1,0,-2)}{(-2)^2+1^2+0^2+(-2)^2}(-2,1,0,-2)\\
    &=&(2a,8a,-3a,2a)
    \end{eqnarray*}
    Note that if $\textbf{v}_1,\textbf{v}_2$ and $\textbf{v}_3$ are linearly dependent, then the length of $\textbf{u}_3$ must be 0, which can only happen when $a=0.$ Thus, for $a=0,$ the required orthogonal basis is $\{\textbf{u}_1,\textbf{u}_2\},$ while for $a\neq0,$ the orthogonal basis is $\{\textbf{u}_1,\textbf{u}_2,\textbf{u}_3\}.$
    \item $\dim(V)=2$ or $\dim(V)=3.$
    \item Firstly, from (a), we have \begin{eqnarray*}
    \textbf{u}_1&=&\textbf{v}_1\\
    \textbf{u}_2&=&2\textbf{v}_1+\textbf{v}_2\\
    \textbf{u}_3&=&-3\textbf{v}_1-\textbf{v}_2+\textbf{v}_3.
    \end{eqnarray*}
    Hence, the transition matrix \textbf{from $T$ to $S$} is given by $\begin{pmatrix}1&2&-3\\0&1&-1\\0&0&1\end{pmatrix}.$ To find the transition matrix from $S$ to $T,$ we only need to find the inverse of the matrix above. Thus, the required matrix is given by $$\begin{pmatrix}1&2&-3\\0&1&-1\\0&0&1\end{pmatrix}^{-1}=\frac{1}{1}\begin{pmatrix}1&0&0\\-2&1&0\\1&1&1\end{pmatrix}^T=\begin{pmatrix}1&-2&1\\0&1&1\\0&0&1\end{pmatrix}.$$
    \item We first find the projection of $(4,7,-9,-5)$ onto $V:=\spn\{(5,2,6,-4),(-12,-3,-12,6),(5,11,3,-4)\}.$ Using the orthogonal basis found in part (a), we have \begin{eqnarray*}
    \text{Proj}_V((4,7,-9,-5))&=&\frac{(5,2,6,-4)\cdot(4,7,-9,-5)}{5^2+2^2+6^2+(-4)^2}(5,2,6,-4)+\frac{(-2,1,0,-2)\cdot(4,7,-9,-5)}{(-2)^2+1^2+0^2+(-2)^2}(-2,1,0,-2)\\
    &+&\frac{(2,8,-3,2)\cdot(4,7,-9,-5)}{2^2+8^2+(-3)^2+2^2}(2,8,-3,2)\\
    &=&(0,9,-3,0).
    \end{eqnarray*}
    Now, we aim to solve the system $$\begin{pmatrix}5&-12&5\\2&-3&11\\6&-12&3\\-4&6&-4\end{pmatrix}\textbf{x}=\begin{pmatrix}0\\9\\-3\\0\end{pmatrix}.$$
    Observe that $-(5,2,6,-4)+(5,11,3,-4)=(0,9,-3,0),$ so a least square solution is given by $\textbf{x}=\begin{pmatrix}-1\\0\\1\\0\end{pmatrix}.$
\end{enumerate}
\section*{Question 3}
\begin{enumerate}[(a)]
    \item We have
    $$
    \begin{pmatrix}
    1&-2&1&3\\1&-1&0&4\\1&0&-1&5
    \end{pmatrix}\xrightarrow[-R_1+R_3]{-R_1+R_3}
    \begin{pmatrix}
    1&-2&1&3\\0&1&-1&1\\0&2&-2&2
    \end{pmatrix}\xrightarrow[-R_2+R_3]{-R_2+R_1}
    \begin{pmatrix}
    1&0&-1&5\\0&1&-1&1\\0&0&0&0
    \end{pmatrix}.
    $$
    Hence, a basis for the row space is given by $\{(1,0,-1,5),(0,1,-1,1)\}.$
    
    On the other hand, a basis for the null space is given by $\{(1,1,1,0)^T, (-5,-1,0,1)^T\}.$
    \item For any matrix $\textbf{A},$ denote the row space and null space of $\textbf{A}$ by $R(\textbf{A})$ and $N(\textbf{A})$ respectively. For any subspace $W,$ define $$W^{\perp}=\{\textbf{v}\in\mathbb{R}^n:\textbf{v}\cdot\textbf{u}=0\ \forall\  \textbf{u}\in W\}.$$
    
    We first show that $(R(\textbf{A}))^{\perp}=N(\textbf{A}).$ Indeed, we have
    \begin{eqnarray*}
    \textbf{u}\in N(\textbf{A}) &\iff& \textbf{A}\textbf{u}=\textbf{0}\\
    &\iff& \text {for any row $\textbf{a}$ of $\textbf{A}$, $\textbf{a}\cdot\textbf{u}=0.$}\\
    &\iff& \textbf{u}\in (R(\textbf{A}))^{\perp}.
    \end{eqnarray*}
    
    Observe that the column space of $\textbf{A}^T$ is equal to the row space of $\textbf{A}.$ Thus, we have $$N(\textbf{B})=R(\textbf{A})\implies R(\textbf{B})=(N(\textbf{B}))^{\perp}=(R(\textbf{A}))^{\perp}=N(\textbf{A}).$$ Hence, it suffices to pick the matrix $\textbf{B}=\begin{pmatrix}-5&-1&0&1\\1&1&1&0\end{pmatrix}.$
    
    For completeness sake, we verify that the matrix $\begin{pmatrix}-5&-1&0&1\\1&1&1&0\end{pmatrix}$ works. Indeed, we have
    $$\begin{pmatrix}-5&-1&0&1\\1&1&1&0\end{pmatrix}\begin{pmatrix}1\\0\\-1\\5\end{pmatrix}=\begin{pmatrix}0\\0\end{pmatrix}\text{ and }\begin{pmatrix}-5&-1&0&1\\1&1&1&0\end{pmatrix}\begin{pmatrix}0\\1\\-1\\1\end{pmatrix}=\begin{pmatrix}0\\0\end{pmatrix}.$$
    The proof is complete.
\end{enumerate}
\section*{Question 4}
\begin{enumerate}[(a)]
    \item We first find the characteristic polynomial of $\textbf{A}.$ We have
    \begin{eqnarray*}
    \det(\textbf{A}-x\textbf{I}_3)&=&\begin{pmatrix}\frac{5}{2}-x&1&-2\\1&1-x&-1\\2&1&-\frac{3}{2}-x\end{pmatrix}\\
    &=&\left(\frac{5}{2}-x\right)\left((1-x)\left(-\frac{3}{2}-x\right)-(-1)\times1\right)-1\left(1\left(-\frac{3}{2}-x\right)-(-1)\times2\right)\\
    &+&(-2)(1\times1-2(1-x))\\
    &=&-x^3+2x^2-\frac{5}{4}x+\frac{1}{4}\\
    &=&-\frac{1}{4}(2x-1)^2(x-1).
    \end{eqnarray*}
    Thus, the eigenvalues of $\textbf{A}$ are $\dfrac{1}{2}$ and $1.$
    \item To find the eigenspace $E_1,$ we solve the system
    \begin{eqnarray*}
    \begin{rowequmat}{ccc|c}
     \frac{3}{2} &  1 & -2 & 0 \\
     1 &  0 & -1 & 0 \\
     2 & 1 & -\frac{5}{2} & 0
    \end{rowequmat}\xrightarrow[2R_3]{2R_1}\begin{rowequmat}{ccc|c}
     3 & 2 & -4 & 0 \\
     1 & 0 & -1 & 0 \\
     4 & 2 & -5 & 0
    \end{rowequmat}\xrightarrow[-R_1+R_3]{-R_2+R_3}\begin{rowequmat}{ccc|c}
     3 & 2 & -4 & 0 \\
     1 & 0 & -1 & 0 \\
     0 & 0 & 0 & 0
    \end{rowequmat}
    \xrightarrow{-3R_2+R_1}\begin{rowequmat}{ccc|c}
     0 & 2 & -1 & 0 \\
     1 & 0 & -1 & 0 \\
     0 & 0 & 0 & 0
    \end{rowequmat}.
    \end{eqnarray*}
    It follows that $E_1=\spn\left\{\begin{pmatrix}2\\1\\2\end{pmatrix}\right\}.$
    \item As for the eigenspace $E_{\frac{1}{2}},$ we have
    $$\begin{rowequmat}{ccc|c}
     2 &  1 & -2 & 0 \\
     1 &  \frac{1}{2} & -1 & 0 \\
     2 &  1 & -2 & 0
    \end{rowequmat}\xrightarrow[-R_1+R_3]{-\frac{1}{2}R_1+R_2}\begin{rowequmat}{ccc|c}
     2 & 1 & -2 & 0 \\
     0 & 0 & 0 & 0 \\
     0 & 0 & 0 & 0
    \end{rowequmat}.$$
    Hence, we have $E_{\frac{1}{2}}=\spn\left\{\begin{pmatrix}1\\0\\1\end{pmatrix},\begin{pmatrix}-1\\2\\0\end{pmatrix}\right\}$
    \item We have $$\begin{pmatrix}2&1&-1\\1&0&2\\2&1&0\end{pmatrix}^{-1}\textbf{A}\begin{pmatrix}2&1&-1\\1&0&2\\2&1&0\end{pmatrix}=\begin{pmatrix}1&0&0\\0&\frac{1}{2}&0\\0&0&\frac{1}{2}\end{pmatrix}=:D.$$
    Then, $$\lim_{n\to\infty}D^n=\begin{pmatrix}1&0&0\\0&0&0\\0&0&0\end{pmatrix}.$$
    
    Since $$\textbf{A}=\begin{pmatrix}2&1&-1\\1&0&2\\2&1&0\end{pmatrix}\begin{pmatrix}1&0&0\\0&\frac{1}{2}&0\\0&0&\frac{1}{2}\end{pmatrix}\begin{pmatrix}2&1&-1\\1&0&2\\2&1&0\end{pmatrix}^{-1},$$ we have
    $$\lim_{n\to\infty}\textbf{A}^n=\lim_{n\to\infty}(\textbf{PDP}^{-1})^n=\lim_{n\to\infty}\textbf{P}\textbf{D}^n\textbf{P}^{-1}=\textbf{P}\left(\lim_{n\to\infty}\textbf{D}^n\right)\textbf{P}^{-1}=P\begin{pmatrix}1&0&0\\0&0&0\\0&0&0\end{pmatrix}P^{-1}=\begin{pmatrix}4&2&-4\\2&1&-2\\4&2&-4\end{pmatrix}.$$
\end{enumerate}
\section*{Question 5}
\begin{enumerate}[(a)]
    \item Clearly, $\textbf{0}\in V.$
    
    For $\textbf{a}=\begin{pmatrix}a_1\\a_2\\a_3\\0\end{pmatrix},\textbf{b}=\begin{pmatrix}b_1\\b_2\\b_3\\0\end{pmatrix}\in V$ and a scalar $c\in\mathbb{R}$ we have $$c\textbf{a}+\textbf{b}=c\begin{pmatrix}a_1\\a_2\\a_3\\0\end{pmatrix}+\begin{pmatrix}b_1\\b_2\\b_3\\0\end{pmatrix}=\begin{pmatrix}ca_1\\ca_2\\ca_3\\0\end{pmatrix}+\begin{pmatrix}b_1\\b_2\\b_3\\0\end{pmatrix}=\begin{pmatrix}ca_1+b_1\\ca_2+b_2\\ca_3+b_3\\0\end{pmatrix}\in V.$$
    Thus, $V$ is a subspace of $\mathbb{R}^4.$
    
    A basis for $V$ is given by $\left\{\begin{pmatrix}1\\0\\0\\0\end{pmatrix},\begin{pmatrix}0\\1\\0\\0\end{pmatrix},\begin{pmatrix}0\\0\\1\\0\end{pmatrix}\right\}.$
    \item $\textbf{A}=\begin{pmatrix}1&0&0\\0&1&0\\0&0&1\\0&0&0\end{pmatrix}.$
    \item $\text{rank}(\textbf{A})=3,$ $\text{nullity}(\textbf{A})=3-\text{rank}(\textbf{A})=3-3=0.$
    \item Take $\textbf{B}=\textbf{A}^T.$ The matrix $\textbf{B}$ is not unique. In fact, for any $\textbf{u}\in\mathbb{R}^3,$ the matrix $(\textbf{I}_3\ \textbf{u})$ satisfies the relation.
    \item Note that $\text{rank}(\textbf{AD})\leq\text{rank}(\textbf{A})=3$ from part (c), but $\text{rank}(\textbf{I}_4)=4,$ which is a contradiction.
\end{enumerate}
\section*{Question 6}
\begin{enumerate}[(a)]
    \item Note that we have $$c_{\textbf{A}}(x)=\det(\textbf{A}-x\textbf{I}_3)=\begin{pmatrix}2-x&a&b\\0&c-x&d\\0&0&e-x\end{pmatrix}=(2-x)(c-x)(e-x)$$ and so $x=2$ is an eigenvalue of $\textbf{A}.$ In particular, $e_1=(1,0,0)^T$ is an eigenvector associated with 2.
    \item By Vieta's formula, the product of roots of the polynomial is $-18.$ It follows that $\dfrac{-18}{2\times9}=-1$ is a root of the polynomial too. Since the characteristic polynomial has three distinct roots, $\textbf{A}$ must be diagonalizable.
\end{enumerate}
\section*{Question 7}
\begin{enumerate}[(a)]
    \item Since $S$ is an orthonormal basis of $V,$ it follows from triangle inequality that \begin{eqnarray*}
    ||\textbf{v}||&=&||c_1\textbf{v}_1+c_2\textbf{v}_2+\cdots+c_k\textbf{v}_k||\\
    &\leq&|c_1|\ ||\textbf{u}_1||+|c_2|\ ||\textbf{u}_2||+\cdots|c_k|\ ||\textbf{u}_k||\\
    &=&|c_1|+|c_2|+\cdots+|c_k|.
    \end{eqnarray*}
    \item Write $$||v||=||c_1\textbf{u}_1+c_2\textbf{u}_2+\cdots+c_k\textbf{u}_k||=\sqrt{c_1^2+c^2_2+\cdots+c_k^2}.$$
    Then, it follows that $||v||^2=c_1^2+c_2^2+\cdots+c_k^2$ and so for each positive integer $1\leq i\leq k,$ we have $c_i^2\leq ||v||^2,$ which implies that $|c_i|\leq ||v||.$ Hence, $|c_i|\leq 1.$
    \setcounter{enumi}{3}
    \item We will prove the result from part (d) only because part (c) follows directly from part (d). From part (b), we have
    \begin{eqnarray*}
    ||\textbf{Av}||&=&||\textbf{A}(c_1\textbf{u}_1+c_2\textbf{u}_2+\cdots+c_k\textbf{u}_k)||\\
    &\leq&|c_1|\ ||\textbf{Au}_1||+|c_2|\ ||\textbf{Au}_2||+\cdots+|c_k|\ ||\textbf{Au}_k||\\
    &\leq&||\textbf{v}||\left(||\textbf{Au}_1+||\textbf{Au}_2||+\cdots+||\textbf{Au}_k||\right)\\
    &=&M||\textbf{v}||.
    \end{eqnarray*}
\end{enumerate}
\end{document}